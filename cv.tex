%%%%%%%%%%%%%%%%%%%%%%%%%%%%%%%%%%%%%%%%%
% Medium Length Professional CV
% LaTeX Template
% Version 2.0 (8/5/13)
%
% This template has been downloaded from:
% http://www.LaTeXTemplates.com
%
% Original author:
% Trey Hunner (http://www.treyhunner.com/)
%
% Important note:
% This template requires the resume.cls file to be in the same directory as the
% .tex file. The resume.cls file provides the resume style used for structuring the
% document.
%
%%%%%%%%%%%%%%%%%%%%%%%%%%%%%%%%%%%%%%%%%

%----------------------------------------------------------------------------------------
%	PACKAGES AND OTHER DOCUMENT CONFIGURATIONS
%----------------------------------------------------------------------------------------

\documentclass{resume} % Use the custom resume.cls style

\usepackage[left=0.75in,top=0.6in,right=0.75in,bottom=0.6in]{geometry} % Document margins
\usepackage[utf8]{inputenc}
\usepackage{lmodern}
%\usepackage{tgpagella}
%\usepackage{fourier}
%\usepackage[sc]{mathpazo}
%\linespread{1.05}         % Palatino needs more leading (space between lines)
\usepackage[T1]{fontenc}

\name{Joe Halliwell} % Your name
\address{47 Lauderdale Street \\ Edinburgh, EH9 1DE} % Your address
\address{joe.halliwell@gmail.com \\ +44~$\cdot$~(0)781~$\cdot$~3124801} % Your phone number and email
\address{Technical leadership \\ System architecture \\ Machine learning \\ Big data}

\begin{document}

%----------------------------------------------------------------------------------------
%	EDUCATION SECTION
%----------------------------------------------------------------------------------------

\begin{rSection}{Education}
\begin{rSubsection}{University of Edinburgh}{1999 -- 2007}{}{}
\item MSc in Artificial Intelligence
\item PhD in Artificial Intelligence
\end{rSubsection}

\begin{rSubsection}{University of Oxford}{1995 -- 2000}{}{}
\item MA in Mathematics and Philosophy
\end{rSubsection}
\end{rSection}

%----------------------------------------------------------------------------------------
%	WORK EXPERIENCE SECTION
%----------------------------------------------------------------------------------------

\begin{rSection}{Experience}

\begin{rSubsection}{Winterwell Associates Ltd}{2008 -- present}
{Winterwell is a machine learning and data analysis consultancy doing a mix of for-hire and internal R\&D.
As a lead consultant, I've worked on dozens of research projects across finance, media, defence, the public sector, education and 
agriculture. As a co-founder, I had special responsibility for business development, finance, partner and client liaison,
devops/system administration and graphic design.}{}
\item Raised £300k in innovation funding, grew to 8 staff
\item Clients include the Home Office, Mott Macdonald, the BBC, Tern Digital, API Software.
\end{rSubsection}

%------------------------------------------------

\begin{rSubsection}{SoDash Ltd}{2010 -- present}
{SoDash is a social media monitoring tool which combines best-in-class coverage of the social web
with a unique text classification engine. SoDash enables large brands to make sense of
social media, segmenting traffic, finding leads and routing messages to the right people.
As a founding director, I've helped design and build the product, plus been responsible for liaising with the commercial team,
finance and devops/system administration.}{}
\item Raised £125k seed capital, grew to 20 staff in Edinburgh, London and San José.
\item Clients include Harrods, Virgin Trains, HSBC, PepsiCo, University of Sunderland, Everest.
\end{rSubsection}

%------------------------------------------------

\begin{rSubsection}{Edinburgh Robotics Ltd}{2005 -- 2008}
{As co-founder and CTO, I lead a wildly ambitious project to develop a complete stack for mobile robotics,
from a Linux-based operating system through a peer-to-peer network protocol and middleware library up to a 3D
simulation environment built on the Open Dynamics Engine.}{} 
\item Core libraries in C, plus C\#, Python and Java bindings, plus amazing documentation.
\item Raised £200k venture funding, £75k SMART innovation award, £125K SPUR commercialisation award.
\end{rSubsection}

\begin{rSubsection}{University of Edinburgh, School of Informatics}{2001 -- 2005}
{I taught several courses at the School of Informatics which helped fund my seemingly interminable thesis write-up. I also devised teaching materials and maintained course websites.}{}
\item Taught AI2 (TA), Knowledge Engineering (Tutor), Knowledge Representation (Tutor), Prolog (Tutor and demonstrator), Third Year Large Practical (Tutor and TA), Operating Systems (TA)
\item Static site generator for the AI2 website and a fun, interactive demonstration of the Festival speech synthesis system.
\end{rSubsection}

\end{rSection}

%\pagebreak

%----------------------------------------------------------------------------------------
%	PROJECTS
%----------------------------------------------------------------------------------------

\begin{rSection}{Recent Projects}
\begin{rSubsection}{Ad server}{2012 - 2013}{Lead developer}{Java, Hadoop, HBase, Play, Bootstrap}
\item High-performance, distributed, rule-based ad server built on the Akka actor framework
\item Self-service management portal for campaign management, reporting, and analytics
\item Probabilistic record linkage store and update algorithm built on Hadoop and HBase
\end{rSubsection}

\begin{rSubsection}{The Thirty-Nine Steps}{2012 - 2013}{Lead developer}{C\#, Unity 3D}
\item John Buchan’s incredible espionage thriller was the debut --- and the first ever --- digital adaptation: a new form of interactive entertainment that merges literature, cinema and gaming. Whether you know The Thirty-Nine Steps inside out, or have never heard of Richard Hannay, this is the classic story as it has never been told before.
\item Platforms: iPad, PC \& OSX boxed, PC, OSX \& Linux Steam; 20,000+ installs
\item See http://thestorymechanics.com/digital-adaptations/the-thirty-nine-steps/
\end{rSubsection}

\begin{rSubsection}{Buddhify}{2012}{Lead developer}{PhoneGap, PHP, Java, Objective-C}
\item The original version of the popular urban meditation app for modern life. Practical, playful and beautifully-designed, Buddhify increases your well-being by teaching you mindfulness-based meditation on the go.
\item Platforms: iPhone, iPad, Android; 15,000+ installs
\item See http://buddhify.com
\end{rSubsection}
\end{rSection}

%----------------------------------------------------------------------------------------
%	TECHNICAL STRENGTHS SECTION
%----------------------------------------------------------------------------------------

\begin{rSection}{Technical skills}

\begin{tabular}{ @{} >{\bfseries}l @{\hspace{6ex}} l }
Programming Languages & Bash, C, C\#, Java, Javascript, PHP, Prolog, Python, Scala, Lisp \\
Data storage & PostgreSQL, MySQL, SQLite, Derby, HBase, ZooKeeper, Hibernate \\
Web frameworks & J2EE, Play, vRaptor, Twisted, Node.js \\
Web client side & HTML, CSS, LESS, jQuery, Bootstrap, AngularJS, HighCharts, D3 \\
Dev tools & git, Ant, Make, Maven, Doxygen, JUnit, Jenkins \\
Data analysis & csvkit, gnuplot, grep, SciKits, Weka \\
Distributed computing & Hadoop MapReduce, HDFS, HBase, YARN, ZooKeeper, Akka \\
Mobile \& gaming & PhoneGap, jQuery Mobile, Unity 3D, iOS, Android \\
Operating System & Linux, GNU userspace, Apache, Cherokee, Ngingx, Exim, Postfix
\end{tabular}

\end{rSection}

%----------------------------------------------------------------------------------------
%	PUBLICATIONS
%----------------------------------------------------------------------------------------

\begin{rSection}{Publications}
J. Halliwell and Q. Shen, {\em Linguistic Probabilities: Theory and Application}, Soft Computing, 2008.\\
J. Halliwell, {\em Linguistic Probability Theory}, PhD These, University of Edinburgh, 2007.\\
J. Halliwell, J. Keppens and Q. Shen, {\em Linguistic Bayesian Networks for Reasoning with
Subjective Probabilities in Forensic Statistics}, Proceedings of the Ninth International
Conference on AI and Law (ICAIL'03), 2003.\\
J. Halliwell and Q. Shen, {\em Towards a Linguistic Probability Theory}, Proceedings of the
11th International Conference on Fuzzy Sets and Systems (FUZZ-IEEE '02), 2002.\\
J. Halliwell and Q. Shen, {\em Towards Temporal Linguistic Variables}, Proceedings of the
10th International Conference on Fuzzy Sets and Systems (FUZZ-IEEE '01), 2001.\\
J. Halliwell and Q. Shen, {\em From fuzzy probabilities to linguistic probability theory},
Proceedings of the 2001 UK Workshop on Computational Intelligence, pp. 129-135,
2001.\\
\end{rSection}

%----------------------------------------------------------------------------------------
%	REFERENCES - AVAILABLE ON REQUEST
% This is going on github, so don't include private addresses
%----------------------------------------------------------------------------------------

%\begin{rSection}{References}
%Professor Qiang Shen, \texttt{qqs@aber.ac.uk},  c/o Department of Computer Science, Penglais, Aberystwyth, Ceredigion, SY23 3DB,Wales, UK \\
%Andrew Back, \textt{arback@computer.org}, 
%Simon Meek, \texttt{email@address.com},
%Danie Winterstein, \textt{daniel@winterwell.com}
%	\end{rSection}

\end{document}
